\chapter{LV 3 am 21.03.2014}
\section{Anforderungsanalyse}
\subsection{Sammeln von Anforderungen}
Um Anforderungen zu sammeln gibt es verschiedene Möglichkeiten:
\begin{itemize}
\item Fragebogen
\item geschlossene Interviews
\item offene Interviews
\end{itemize}

Sehr gut geeignet ist eine Mischung aus allen drei Varianten. Wichtig ist, dass man die Leute interagieren lässt - sprich man benötigt ein gewisses Maß an Flexibilität und muss sehr gut zuhören können. 
\\\\
Bei der Sammlung von Anforderungen treten einige Probleme auf. Eine sehr große Hürde stellt der Jargon und das implizite Wissen dar. Die Leute gehen davon aus, dass jeder ihren Jargon (mit allen Begrifflichkeiten) kennt und versteht. Das implizite Wissen ist den Leuten meistens nicht bewusst (z.B. aus Gewohnheit) und wird von ihnen zurückgehalten. Beim nicht expliziten Wissen, wissen die Leute, dass sie es haben und gehen davon aus, dass jeder dieses Wissen hat, da es für sie selbstverständlich ist. 
\\\\
Wichtig ist, dass man nicht sofort mit Interviews anfängt, sondern sich erst in die Domäne einliest. Andernfalls ist das Ergebnis von den Interviews nicht zu gebrauchen, da einfach das Verständnis für die Domäne nicht vorhanden ist.
\\\\
Szenarien stellen eine weitere Technik zur Sammlung von Anforderungen dar. Es ist zu empfehlen bei der Erstellung von Szenarien Ad-hoc vorzugehen und zur Dokumentation UML-Diagramme zu verwenden. Wichtig dabei ist es auch auf die Details zu achten, um das Gesamtbild besser verstehen zu können.
\\\\
Eine weitere Technik zur Sammlung von Anforderungen ist die Ethnografische Methoden. Durch diese Variante kann man durch Beobachten von Personen Anforderungen herausgefunden. Dabei wird diese Person neutral und ohne jegliche Interaktion mit dieser beobachtet. Es müssen stets stillschweigend Notizen gemacht werden und am Ende wird mit einem Experten das Beobachtete diskutiert. Es ist von Vorteil dies  mit der jeweiligen beobachteten Person zu machen und nicht mit deren Chef, da sonst weitere Beobachtungen durch wenig Kooperation der Mitarbeiter behindert werden könnten. 
\\
Durch diese Variante kann man sehr gut implizite, nicht explizite Anforderungen und Abweichungen erkennen. Auch die Anforderungen der Benutzer können dadurch sehr gut ermittelt werden. 

\section{Klassifizierung von Anforderungen}
Bei der Klassifizierung hat man die Aufgabe die Struktur der Anforderungen aus den unterschiedlichen Quellen zu definieren. 
Es ist wichtig Duplikate sowie Synonyme zu erkennen, die Beziehungen zwischen den Anforderungen zu definieren und eine Gruppierung der Anforderungen durchzuführen.

\section{Validierung von Anforderungen}
Es gilt zu prüfen, ob die Anforderungen gültig (filtern von falschen und unnötigen Anforderungen), konsistent, vollständig (sehr schwer zu prüfen), realisierbar (genügt das Budget, Zeit? ist die Technologie geeignet?) und verifizierbar sind.
\\\\
Um dies Überprüfung durchzuführen kann man ein Anforderungsreview machen, einen Prototypen entwickeln oder Testfälle erzeugen.
\\\\
Für ein Anforderungsreview benötigen wir aus allen wichtigen Gruppen der Stakeholder einen Vertreter, einen Systemarchitekten und einen Vertreter der Softwareentwickler (diese müssen am Ende ja wissen, was sie entwickeln sollen). Die Durchführung erfolgt durch den jeweiligen Anwendervertretern. Es werden dabei alle Anforderungen diskutiert. Durch das Anforderungsreview können Konflikte, Widersprüche, Fehler und Versäumnisse ermittelt werden. Sie sollten in einem Review-Bericht festgehalten werden. Die Prüfung auf Konsistenz und Vollständigkeit fallen unter die Kategorie der Notwendigen Prüfungen. Prüfungen bezüglich der Verifizierbarkeit, Verständlichkeit sowie Anpassungsfähigkeit sind Optionale Prüfungen.

\section{Priorisierung von Anforderungen}
Die Priorisierung ist speziell bei einer agilen Vorgehensweise sehr wichtig. 
\\\\
Der Grundgedanke ist durch Phasenweise Implementierung das Leben der Anwender durch kontinuierliche Softwareupdates zu erleichtern. Es minimiert auch das Risiko der Einführung der Software.
\\\\
Die Prioritätssteuerung erfolgt über eine Bewertungsformel. Es wird jeweils eine Bewertung zu jeder einzelnen Anforderung durch das Nutzwertanalyseteam und das Kostenanalyseteam erstellt.  Beide Teams bewerten völlig unabhängig voneinander und ohne Kommunikation zwischen den Teams. Nach der Auswertung erhält man eine Liste der Anforderungen welche für die nächste Ausbaustufe abgearbeitet werden sollten. 

\section{Dokumentation und Verwaltung}
\subsection{Modelle}
Bieten eine detaillierte und formalisierte Dokumentation über den Ist-Zustands und Soll-Zustands. Es werden auch Sichten von Kunden oder anderen Abteilungen auf beide Zustände festgehalten. 
\subsubsection{Kontextmodelle}
Mit diesem Modell werden die Systemgrenzen hinsichtlich des Gesamtsystems definiert. Hier ist es wichtig den Kontext zu definieren. Neben dem Gesamtsystem werden die technischen Systeme betrachtet. Bei den technischen Systemen ist die Definition des Scopes besonders wichtig. Wichtig bei der Erstellung eines Kontextmodelles die die Auswahl der Verfügbaren Notationen (UseCase-Diagramm, Architekturdiagramm, etc.).

\subsubsection{Verhaltensmodelle}
Es werden Daten- und Ereignis-orientiert die Abläufe im System definiert. Es gibt auch Mischformen (z.B. eine Banküberweisung). Zur Veranschaulichung dienen Diagramme wie Datenflussdiagramme, Prozessdiagramme, Zustandsdiagramme, Sequenzdiagramme oder Szenarios. 

\subsubsection{Datenmodelle}
Für die Darstellung der Datenmodelle gibt es ER-Diagramme, UML-Klassendiagramme, XML-Schema- OWL und diverse andere Diagrammtypen. Sie dienen zur logischen Definition der Datenstrukturen hinsichtlich transients und persistents. Auch hier ist die Auswahl der verfügbaren Notation von großer 
Wichtigkeit.
\\\\
Ein wichtiges Werkzeug bei den Datenmodellen ist das Data Dictionary. Hier speichert man sich die Klassen und Attribute. Man speichert sich zu den Entitäten gewisse Eigenschaften wie zum Beispiel den Namen, Beschreibung und Typ.

\subsubsection{Objektorientierte Modellierung}
Bei der Objektorientierten Modellierung geht es um die Vereinigung der Funktionalität von Daten und Verhaltensmodellierung. Hier ist die Rede von Daten, Datenflüssen sowie Ereignissen. Als Standard Notation wird hierbei UML verwendet. Bei der Modellierung werden folgende Highlights verwendet.
\begin{itemize}
\item Klassifikationshierarchien
\item Aggregationsbeziehungen
\item Operationen (Objektverhalten)
\end{itemize}

\subsubsection{Strukturierte Methoden}
Die strukturierte Methode beschreibt eine detailliert definierte Vorgehensweise bei der Software-Entwicklung. Normalerweise basierend auf einem Satz von Diagrammtypen in Verbindung mit Regeln und Richtlinien. Beispiele für Strukturierte Methoden sind:
\begin{itemize}
\item JSP
\item V-Modell
\item RUP
\end{itemize}

Durch den Einsatz von strukturierten Methoden wird gewährleistet, dass immer eine gewisse Qualität erreicht wird. 
\\\\
Die Nachteile sind mangelnde Unterstützung nicht-funktionaler Anforderungen und die Anwendbarkeit für konkrete Probleme. Letzteres benötigt sehr viel Erfahrung um die Entscheidung \textit{Wann verwende ich welche Methode} korrekt treffen zu können. Es muss auch beachtet werden, dass die Anpassung einer Problemklasse sehr schwierig ist und dass ein enormer Dokumentationsaufwand oft das Wesentliche verdeckt. Eine weitere Gefahr bei dieser Methoden ist der hohe  Detaillierungsgrad, welcher das Verständnis für das System erschwert.

\subsection{Anforderungsmanagementsystem}
Erfahrungsgemäß sind die Anforderungen am Anfang eines Projektes stets unvollständig und sie ändern sich im laufe eines Projektes fortlaufend. Die Anforderungen helfen allerdings dabei das Problem bzw. die Probleme besser zu verstehen. Deshalb ist der Umgang mit Anforderungen und ihren Änderungen von enormer Bedeutung.  Hier ist es wichtig auf eine gute Dokumentation zu achten und die richtigen Werkzeuge zur Unterstützung einzusetzen. Vor allem die Aufgabenübersicht (Anforderungssammlung, Beziehungsmanagement, etc) ist nicht zu vernachlässigen.
\\\\
Man muss bei den Anforderungen zwischen verschiedenen Typen unterscheiden. Es gibt dauerhafte Anforderungen, die relativ stabil sind und mit dem Kernbereich des Systems verbunden. Sie können aus standardisierten Modellen des Anwendungsbereichs abgeleitet werden. Daneben gibt es veränderliche Anforderungen mit einer sehr hohen Änderungswahrscheinlichkeit. Es müssen wirtschaftliche, technische und gesetzliche Randbedingungen beachtet werden. 