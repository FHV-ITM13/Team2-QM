\chapter{LV 3 am 21.03.2014}
TODO Überschriften kontrollieren und anpassen an die Folien
\section{Anforderungsanalyse}
TODO REMOVE: Start Seite 46
\subsection{Sammeln von Anforderungen}
Um Anforderungen zu sammeln gibt es verschiedene Möglichkeiten:
\begin{itemize}
\item Fragebogen
\item geschlossene Interviews
\item offene Interviews
\end{itemize}

Sehr gut geeignet ist eine Mischung aus allen drei Varianten. Wichtig ist, dass man die Leute interagieren lässt - sprich man benötigt ein gewisses Maß an Flexibilität und muss sehr gut zuhören können.
\\\\
Bei der Sammlung von Anforderungen treten einige Probleme auf. Eine sehr große Hürde stellt der Jargon und das implizite Wissen dar. Die Leute gehen davon aus, dass jeder ihren Jargon (mit allen Begrifflichkeiten) kennt und versteht. Das implizite Wissen ist den Leuten meistens nicht bewusst (z.B. aus Gewohnheit) und wird von ihnen zurückgehalten. Beim nicht expliziten Wissen, wissen die Leute, dass sie es haben, und gehen davon aus, dass jeder dieses Wissen hat, da es für sie selbstverständlich ist. 
\\\\
Wichtig ist, dass man nicht sofort mit Interviews anfängt, sondern sich erst in die Domäne einliest. Andernfalls ist das Ergebnis von den Interviews nicht zu gebrauchen.
\\\\
Szenarien (Seite 47)
Es ist zu empfehlen bei der Erstellung von Szenarien Ad-hoc vorzugehen und zur Dokumentation UML-Diagramme verwenden. Es kann schon sehr früh mit der detaillierten Dokumentation begonnen werden. Details sind wichtig um alles korrekt verstehen zu können.
\\\\
Ethnografische Methoden (S. 48)
Durch diese Variante kann man durch Beobachten der Leute Anforderungen herausgefunden. Es müssen stets stillschweigend Notizen gemacht werden und am Ende wird mit einem Experten das Beobachtete diskutiert. Dies sollte man mit der jeweiligen beobachteten Person machen und nicht mit deren Chef, da sonst weitere Beobachtungen durch wenig Kooperation der Mitarbeiter behindert werden. 
\\
Durch diese Variante kann man sehr gut implizite, nicht explizite Anforderungen und Abweichungen erkannt werden. Auch die Anforderungen der Benutzer können sehr gut ermittelt werden. 
\section{Klassifizierung von Anforderungen}
???
\section{Validierung von Anforderungen}
Es gilt zu prüfen, ob die Anforderungen gültig (filtern von falschen und unnötigen Anforderungen), konsistent, vollständig (sehr schwer zu prüfen), realisierbar (genügt das Budget, Zeit? ist die Technologie geeignet?) und verifizierbar sind.
\\\\
Um dies Überprüfung durchzuführen kann man ein Anforderungsreview machen, einen Prototypen entwickeln oder Testfälle erzeugen.
\\\\
Für ein Anforderungsreview benötigen wir aus allen wichtigen Gruppen der Stakeholder einen Vertreter, einen Systemarchitekten und einen Vertreter der Softwareentwickler (müssen am Ende ja wissen, was sie entwickeln sollen). Die Durchführung wird von den jeweiligen Anwendervertretern durchgeführt und alle Anforderungen diskutiert. Durch das Anforderungsreview können Konflikte, Widersprüche, Fehler und Versäumnisse ermittelt werden. Sie sollten in einem Review-Bericht festgehalten werden. 

\section{Priorisierung von Anforderungen}
Die Priorisierung ist speziell bei einer agilen Vorgehensweise sehr wichtig. 
\\\\
Der Grundgedanke ist durch Phaseweise Implementierung das Leben der Anwender durch kontinuierliche Softwareupdates zu erleichtern. (?????). Es minimiert auch das Risiko der Einführung der Software.
\\\\
Die Prioritätsfestlegung erfolgt über eine Bewertungsformel. Es wird jeweils eine Bewertung zu jeder einzelnen Anforderung durch das Nutzwertanalyseteam und das Kostenanalyseteam. Beide Teams bewerten völlig unabhängig voneinander und ohne Kommunikation zwischen den Teams. Nach der Auswertung erhält man eine Liste der Anforderungen für die nächste Ausbaustufe. 
\section{Dokumentation und Verwaltung}
\subsection{Modelle}
Bieten eine detaillierte und formalisierte Dokumentation über des Ist-Zustands und Soll-Zustands. Es werden auch Sichten von Kunden oder anderen Abteilungen auf beide Zustände festgehalten. 
\subsubsection{Kontextmodelle}
Mit diesem Modell werden die Systemgrenzen hinsichtlich des Gesamtsystems definiert. Hier ist es wichtig den Kontext zu definieren. Neben dem Gesamtsystem werden die technischen .....

TODO ZU LANGSA;!! SEITE 64

Scope: was man wirklich implementieren will
Kontext: das ganze um den Scope herum.
\subsubsection{Verhaltensmodelle}
Es wird daten- und ereignis-orientiert die Abläufe im System definiert. Es gibt auch Mischformen (z.B. eine Banküberweisung). Zur Veranschaulichung dienen Diagramme wie Datenflussdiagramme, Prozessdiagramme, Zustandsdiagramme, Sequenzdiagramme oder Szenarios. 

\subsubsection{Datenmodelle}
Für die Darstellung der Datenmodelle gibt es ER-Diagramme, UML-Klassendiagramme, XML-Schema- OWL und weitere. Sie dienen zur logischen definition der Datenstrukturen hinsichtlich transient, persistent...
\\\\
TODO Seite 76.
\\\\
Ein wichtiges WErkzeug ist das Data Dictionary. Hier speichert man sich die Klassen und Attribute...
Man speichert sich zu den Entitäten gewisse Eigenschaften wie zum Beispiel den Namen, Beschreibung und Typ.
\\\\
TODO Seite 77.
\\\\

\subsubsection{Strukturierte Methoden}
TODO Seite 89, 90, 91?
\\\\
Durch den Einsatz von strukturierten Methoden wird gewährleistet, dass immer eine gewisse Qualität erreicht wird. 
\\\\
Die Nachteile sind mangelde Unterstütztung nicht-funktionaler Anforderungen und die Anwendbarkeit für konkrete Probleme. Letzteres benötigt sehr viel Erfahrung um die Entscheidung \textit{Wann verwende ich welche Methode} korrekt treffen zu können. Es muss auch beachtet werden, dass die Anpassung einer Problemklasse sehr schwierig ist und dass ein enormer Dokumentationsaufwand oft das Wesentliche verdeckt. Eine weitere Gefahr bei strukturirten Methoden ist der hohe Grad an Detailliertheit (...? Seite 92).

\subsection{Anforderungsmanagementsystem}
Erfahrungsgemäß sind die Anforderungen am Anfang eines Projektes stets unvollständig und sie ändern sich immer wieder. 
\\\\
Todo: Folie 97
\\\\
Man muss bei den Anforderungen zwischen verschiedenen Typen unterscheiden. Es gibt dauerhafte Anforderungen, die relativ stabil sind und mit dem Kernbereich des Systems verbunden. Sie können aus standardisierten Modellen des Anwendungsbereichs abgeleitet werden. Daneben gibt es veränderliche Anforderungen mit einer sehr hohen Änderungswahrscheinlichkeit. Es müssen wirtschaftliche, technische und gesetzliche Randbedingungen beachtet werden. 
\\\\
Todo: Seite bis 107