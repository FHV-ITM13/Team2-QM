\chapter{LV 2 am 07.03.2014}
Zu Beginn wurden kurz die Lösungen der Hausaufgabe besprochen. Es wurde verdeutlicht, dass sogar so eine einfache Aufgabenstellung von einigen Studenten anders implementiert wurde, als ursprünglich gedacht.
\\\\
Aufbauend auf LV 1 wurden wichtige Begriffe wie Software, Software-Engineering, Softwareprozess usw. definiert. Die wichtigsten Definitionen, auch von zukünftigen Vorträgen, sind im Glossar angeführt. Dabei wurde auch das Vorgehensmodell und die Methode des Software-Engineerings vorgestellt.
\\\\
Bei der Methode des Software-Engineerings geht es um einen strukturierten Ansatz für die Software-Entwicklung mit Ziel eine möglichst hohe Qualität der Software bei gleichzitig geringen Entwicklungskosten zu erreichen.
\\\\
Das Vorgehensmodell definiert wie der Softwareentwicklungsprozess abläuft. Bekannte Modelle sind dabei Scrum, Wasserfall oder das V-Modell. Dabei stellt sich heute oft die Frage, ob man sich für einen klassisches oder agiles Vorgehensmodell entscheidet. Der größte Unterschied zwischen agil und klassisch ist, dass beim Agilen Ansatz der Kunden viel mehr involviert ist und öfters das Feedback von ihm in die Entwicklung mit einfließt. Auch sind die Iterationen beim Agilen Vorgehen meist kürzer.
\\\\
Anschließend behandelten wir das Thema der soziotechnischen Systeme. Dabei wurde der Begriff genauestens definiert, die wesentlichen Eigenschaften und der Lebenszyklus sowie die Umfeldfaktoren vorgestellt.
\\\\
Nach den soziotechnischen Systemen behandelten wir das Thema der kritischen Systeme. Dabei geht es um Systeme dessen Ausfall oder Fehlfunktion sich auf die Umgebung in Form von z.B. wirtschaftlichen Verlusten, Schäden an der Umwelt  auswirken. Wichtige Faktoren dabei sind die Verfügbarkeit, Zuverlässigkeit, Betriebssicherheit und Systemsicherheit. Der Sammelbegriff für diese Punkte heißt Verlässlichkeit.
\\\\
Zum Schluss der Lehrveranstaltung wurden die ethnischen Herausforderungen beim Software-Engineering behandelt. Dabei geht es um Vertraulichkeit gegenüber dem Arbeitgebern und Kunden, um den Schutz von geistigem Eigentum und den Schutz vor Computermissbrauch. Es gibt auch einen Knigge für das professionelle Verhalten des Software-Engineering definiert durch ACM/IEEE.

\section{Anforderungsanalyse}
Bei der Anforderungsanalyse ist es wichtig die Projektbeteiligten eindeutig zu identifizieren. Hierbei ist es auch wichtig deren Erwartungen festzustellen. Sprachbarrieren zu beseitigen sowie die Unterscheide im Bezug auf Anforderungen zu aufzuklären. Anbei sehen sie einige Beispiele solcher Beteiligten. 
\begin{itemize}
\item Endanwender
\item Kooperationspartner
\item Systemadministratoren
\item Sicherheitsbeauftragte
\item Marketingabteilung andere betroffene Mitarbeiter
\item Hard- und Softwareentwickler Manager
\item Technologie-Experten
\item Anwendungsexperten
\item Betriebsratsmitglieder
\item Behörden
\end{itemize}

