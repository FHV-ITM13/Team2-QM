\chapter{Glossar}
\begin{itemize}
\item 	Aggregationsbeziehungen\\
Falls Objekte einer Klasse Bestandteil von einem anderen Objekt sind, liegt eine Aggregationsbeziehung vor.
\item 	Anforderung\\
Eine Anforderung ist eine Aussage über die notwendige Beschaffenheit oder Fähigkeit, die ein System erfüllen oder besitzen muss, um einen Vertrag zu erfüllen oder einer Norm, einer Spezifikation oder anderen, formell vorgegebenen Dokumenten zu entsprechen.
\item 	Anforderungsreview\\
Bei einem Anforderungsreview werden die Anforderungen von allen Stakeholdern sowie Systemarchitekten und Softwareentwickler gemeinsam diskutiert. Dadurch werden Konflikte, Fehler und Versäumnisse aufgedeckt. Es dient der Validierung der Anforderungen.
\item Anomalie \\ Anomalie (Plural: die Anomalien) ist ein Fremdwort aus dem Griechischen und bedeutet „Unregelmäßigkeit“.
\item 	Architekturdiagramm\\
Bildet die Architektur einer Software ab.

\item Ausgabefehler \\ Eine Fehlerklasse. (z.B. zweimal ausgegebene Variable ohne zwischenzeitliche Zuweisung)
\item 	Betriebssicherheit \\ Betriebssicherheit ist die Eigenschaft eines System, bei richtiger wie falscher Benutzung und auch beim Auftreten externer Fehler kein Verhalten zu zeigen, das Benutzer oder Dritte schädigt oder  gefährdet. geht weiter als Zuverlässigkeit P07.pdf S 42
\item 	Blackbox-Test (Tests.pdf S. 14)\\
Test gegen die Spezifikation ohne Kenntnis des Codes.
\item 	Bug\\
Bezeichnet ein Fehlverhalten von Computerprogrammen.
\item 	Client/Server-Architektur\\
Unter der Client-Server-Architektur versteht man eine kooperative Informationsverarbeitung, bei der die Aufgaben zwischen Programmen auf verbundenen Rechnern aufgeteilt werden.
\item 	Data Dictionary\\
Ein Katalog von Metadaten, der die Definitionen und Darstellungsregeln für alle Anwendungsdaten und die Beziehungen zwischen den verschiedenen Datenobjekten enthält, damit der Datenbestand redundanzfrei und einheitlich strukturiert wird. Es ist ein Anwendungsfall eines spezifischen Datenmodells.
\item Datenfehler \\ Eine Fehlerklasse. (z.B. uninitialisierte Variablen)
\item 	Dienstorientierte Architekturen\\
Serviceorientierte Architektur (SOA, englisch service-oriented architecture), auch dienstorientierte Architektur, ist ein Architekturmuster der Informationstechnik
\item Eingabefehler  \\ Eine Fehlerklasse. (z.B. zweimal ausgegebene Variable ohne zwischenzeitliche Zuweisung)
\item 	Emulator\\
Nachbildung wesentlicher Verhaltensaspekte eines Systems durch ein anderes.
\item 	Endknoten\\
Knoten ein einem Graph von dem man nicht mehr weg kommt.
\item 	endliche Automaten\\
Ein Modell eines Verhaltens, bestehend aus Zuständen, Zustandsübergängen und Aktionen.
\item 	Entwickler\\
Als Entwickler wird die Person bezeichnet die die Software entwickelt.
\item 	ER-Diagramm\\
Die Darstellung eines Entity-Relationship-Modells
\item 	Ereignis\\Unter einem Ereignis (englisch event) versteht man in der ereignisorientierten Programmierung eine Begebenheit, die – über ein Event-Handler-Programm – eine Aktion und (in deren Folge) ggf. eine Zustandsveränderung auslöst. Diese Ereignisse können Benutzereingaben (Mausklick, Taste, Spracheingabe, Geräteanschluss, …) oder Systemereignisse (Zeitpunkt, Fehler, Datenveränderung, Sensor, …) sein.
\item 	ESSENCE Kernel\\
Ein Bereich aus der SEMAT Initiative
\item 	Funktionale Anforderungen\\
Eine funktionale Anforderung legt fest, was das Produkt tun soll. Ein Beispiel:
\\
„Das Produkt soll den Saldo eines Kontos zu einem Stichtag berechnen.“
\item 	Hardware\\
Die mechanische und elektronische Ausrüstung eines Systems.
\item Heuristiken \\ Heuristik bezeichnet ein analytisches Vorgehen, bei dem mit begrenztem Wissen über ein System mit Hilfe von mutmaßenden Schlussfolgerungen Aussagen über das System getroffen werden.
\item 	Informatik\\Informatik ist die „Wissenschaft der systematischen Verarbeitung von Informationen, insbesondere der automatischen Verarbeitung mit Hilfe von Digitalrechnern“.
\item 	JSP\\Jackson structured programming (JSP) ist eine Methode zum strukturierten Entwurf von Computerprogrammen.
\item 	Kunden\\Ein Kunde ist eine Person oder eine Institution, die ein offensichtliches Interesse am Vertragsschluss zum Zwecke des Erwerbs eines Produkts oder einer Dienstleistung gegenüber einem Unternehmen oder einer Institution zeigt.
\item 	Manager\\
Ein Manager ist eine Person im Anstellungsverhältnis, welche Managementaufgaben in einer Organisation wahrnimmt. Die wichtigsten Managementaufgaben sind Planung, Organisation, Führung und Kontrolle.
\item 	mandatory requirements\\
Was muss das fertige System können
\item 	Marketingstrategen
\\Personen die Marketingstrategien umsetzen.
\item 	Mehrprozessor-Architekturen\\
Computersysteme mit zwei oder mehr Prozessoren
\item 	Messwert\\Ein Messwert ist der Wert einer Messgröße, der von einem Messgerät oder einer Messeinrichtung geliefert wird.
\item 	Nicht-Funktionale Anforderungen\\
Eine nichtfunktionale Anforderung (englisch non-functional requirement, NFR) legt fest, welche Eigenschaften das Produkt haben soll. Ein Beispiel:
\\
„Das Produkt soll dem Anwender innerhalb von einer Sekunde antworten.“
\item 	Normen\\Normen sind anerkannte Regeln in der Technik. 
\item 	Objektorientierten Modellierung\\Die objektorientierte Modellierung bildet einen Ansatz zur Analyse und Entwicklung von Systemen, der wesentlich auf den Konzepten „Objekt“, „Klasse“ und „Vererbung“ beruht.
\item 	OWL\\
Die Web Ontology Language (kurz OWL) ist eine Spezifikation des World Wide Web Consortiums (W3C), um Ontologien anhand einer formalen Beschreibungssprache erstellen, publizieren und verteilen zu können. Es geht darum, Termini einer Domäne und deren Beziehungen formal so zu beschreiben, dass auch Software die Bedeutung verarbeiten („verstehen“) kann.
\item 	Peer-to-Peer-Architekturen\\
P2P ist ein netzwerkbasiertes Modell für Anwendungen, wo Computer Ressourcen und
Dienste über den direkten Austausch teilen. Alle Computer können Dienste beanspruchen
oder zur Verfügung stellen, worauf besonders acht darauf gelegt wird, dass keine
Abhängigkeit von zentralen Servern besteht.
\item 	persistent\\Persistenz ist in der Informatik der Begriff, der die Fähigkeit bezeichnet, Daten (oder Objekte) oder logische Verbindungen über lange Zeit (insbesondere über einen Programmabbruch hinaus) bereitzuhalten
\item 	Pflichtenheft\\Das Pflichtenheft beschreibt in konkreter Form, wie der Auftragnehmer die Anforderungen des Auftraggebers zu lösen gedenkt – das sogenannte wie und womit.
\item 	Prozessdiagramm\\Die Darstellung eines Prozessablauf.
\item 	Qualität\\
Qualität wird laut der Norm EN ISO 9000-2005 (der gültigen Norm zum Qualitätsmanagement), als Grad, in dem ein Satz inhärenter Merkmale Anforderungen erfüllt, definiert. Die Qualität gibt damit an, in welchem Maße ein Produkt (Ware oder Dienstleistung) den bestehenden Anforderungen entspricht.
\item 	Qualitative Anforderung: es hat die Eigenschaft oder hat sie nicht
\item 	Qualitätsmanagement\\
Qualitätsmanagement (QM) bezeichnet alle organisatorischen Maßnahmen, die der Verbesserung der Prozessqualität, der Leistungen und damit den Produkten jeglicher Art dienen. 
\item 	Quantitative Anforderung: ist eine messbare Eigenschaft z.B. die Antwortzeit beträgt max. 1 Sekunde
\item 	redundante Hardware\\
Bei Hardware heit Redundanz, dass mehrere Geräte vorhanden sind (obwohl das System auch mit einem Bruchteil auskommen würde), die zu mehr Stabilität oder Geschwindigkeit führen. z.B. RAID
\item 	Reengineering\\
In diesem Dokument bedeutet Reengineering die Umwandlung eines alten System in ein neues, ganz besonders im Blick auf die verwendete Programmiersprache.
\item 	Regressionstest \\
Test, der neben der erwarteten Funktionalität sicherstellen soll, dass die Software frei ist von allen bislang bereits beseitigten Fehlern
\item 	Review\\
Mit dem Review werden Arbeitsergebnisse der Softwareentwicklung manuell geprüft. Jedes Arbeitsergebnis kann einer Durchsicht durch eine andere Person unterzogen werden. Der oder das Review ist eine statische Testmethode und gehört in die Kategorie der analytischen Qualitätssicherungsmaßnahmen.

In Anlehnung an die IEEE-Norm 729 ist das Review ein mehr oder weniger formal geplanter und strukturierter Analyse- und Bewertungsprozess, in dem Projektergebnisse einem Team von Gutachtern präsentiert und von diesem kommentiert oder genehmigt werden.
\item 	Risikomanagement\\
Risikomanagement umfasst sämtliche Maßnahmen zur systematischen Erkennung, Analyse, Bewertung, Überwachung und Kontrolle von Risiken.
\item 	ROCOF \\ Rate Of Occurrence Of Failures \\Die Ausfallrate ist eine Kenngröße für die Zuverlässigkeit eines Objektes. Sie gibt an, wie viele Objekte in einer Zeiteinheit durchschnittlich ausfallen.
\item 	RUP\\Der Rational Unified Process (RUP) ist ein kommerzielles Produkt der Firma Rational Software, die seit 2003 Teil des IBM-Konzerns ist. Es beinhaltet sowohl ein Vorgehensmodell zur Softwareentwicklung als auch die dazugehörigen Softwareentwicklungsprogramme. IBM entwickelt den RUP und die zugehörige Software weiter. Die 9. Version ist die derzeit (2006) aktuelle Version. Der RUP benutzt die Unified Modeling Language (UML) als Notationssprache. Der RUP wurde von Philippe Kruchten in seiner Urform erstmals 1998 vorgestellt.
\item Prozess \\ Ein Satz von in Wechselbeziehungen stehenden Aktivitäten und Ressourcen, die Eingaben in Ergebnisse umgestalten. [ISO 12207]
\item 	Sicherheitslücke\\Eine Sicherheitslücke ist im Gebiet der Informationssicherheit ein Fehler in einer Software, durch den ein Programm mit Schadwirkung (Malware) oder ein Angreifer in ein Computersystem eindringen kann.
\item 	Simulator\\Vereinfachtes Modell eines Systems zum Zweck der Analyse dieses Systems.
\item 	Software\\
Software ist ein Sammelbegriff für ausführbare Programme und den dazu zugehörigen Daten. Sie dient dazu, Aufgaben zu erledigen, indem sie von einem Prozessor ausgewertet wird und so Software gesteuerte Geräte in ihrer Arbeit beeinflusst.
\item 	Softwarearchitekten \\
Ein Begriff für Menschen die im Bereich der Softwaretechnik an der Softwarearchitektur und deren Implementierung arbeiten.
\item 	Software-Engineering\\
Beschäftigt sich mit der Herstellung bzw. Entwicklung von Software, der Organisation und Modellierung der zugehörigen Datenstrukturen und dem Betrieb von Softwaresystemen
\item 	Softwareprozess\\Software-Prozesse beschreiben den Ablauf der Entwicklung und der Pflege von Software.
\item 	Software-Validierung \\  Ob die Software für den Kunden einen Wert hat
\item 	Software-Verifizierung \\ Übereinstimmung der Software mit den Spezifikationen
\item 	Soziotechnisches System\\Unter einem soziotechnischen System versteht man eine organisierte Menge von Menschen und Technologien, welche in einer bestimmten Weise strukturiert sind, um ein spezifisches Ergebnis zu produzieren.
\item 	Spezifikation\\Eine Spezifikation  ist eine formalisierte Beschreibung eines Produktes, eines Systems oder einer Dienstleistung. Ziel der Spezifikation ist es, Merkmale zu definieren und zu quantifizieren (Toleranzwerte), mit denen das Werk oder die Dienstleistung des Auftragnehmers bei der Übergabe an den Auftraggeber bzw. Käufer geprüft und durch den Auftraggeber abgenommen werden kann, bzw. nach der der Auftragnehmer bzw. Verkäufer die Bezahlung fordern kann, wenn die Merkmale der Spezifikation erreicht wurden. Die Spezifikation enthält in der Regel für jede spezifizierte Eigenschaft eine präzise Referenz zu der Anzuwendenden Prüfmethode für das jeweilige Merkmal.
\item 	Stakeholder\\
Als Stakeholder wird eine Person oder Gruppe bezeichnet, die ein berechtigtes Interesse am Verlauf oder Ergebnis eines Prozesses oder Projektes hat.
\item 	Standards\\Ein Standard ist eine vergleichsweise einheitliche oder vereinheitlichte, weithin anerkannte und meist angewandte (oder zumindest angestrebte) Art und Weise, etwas herzustellen oder durchzuführen, die sich gegenüber anderen Arten und Weisen durchgesetzt hat. 
\item 	Startknoten \\Der Anfangsknoten in einem Graphen
\item 	Strategie\\
Strategie ist ein längerfristig ausgerichtetes Anstreben eines Ziels unter Berücksichtigung der verfügbaren Mittel und Ressourcen.
\item Steuerungsfehler \\ Eine Fehlerklasse. ((z.B. unerreichbarer Code)
\item Speicherverwaltungsfehler \\ Eine Fehlerklasse. ((z.B. nicht zugewiesene Zeiger)
\item 	SuT\\System Under Test (SUT) bezeichnet das zu testende System in einem Testszenario.
\item 	Systemarchitekten\\Dem Systemarchitekten kommt die zentrale Rolle für Systementwurf und -spezifikation zu. Er entwirft auf Basis der Gesamtsystemspezifikation (Pflichtenheft) die Systemarchitektur.
\item 	Systemgrenzen\\Die Grenze trennt, was zum System gehört von dem, was nicht zum System gehört
\item 	Systemverfügbarkeit \\Die Verfügbarkeit eines technischen Systems ist die Wahrscheinlichkeit oder das Maß, dass das System bestimmte Anforderungen zu bzw. innerhalb eines vereinbarten Zeitrahmens erfüllt. Sie ist ein Qualitätskriterium und eine Kennzahl eines Systems. Die Verfügbarkeit lässt sich anhand der Zeit, in der ein System verfügbar ist, definieren:

\item 	Systemwiederherstellungszeit \\  Die Zeit die das System benötigt um nach einer Störung wieder Funktionsfähig zu sein.
\item 	Systemzuverlässigkeit \\Die Systemzuverlässigkeit beschreibt die Funktionssicherheit, Verfügbarkeit und Wartungsfähigkeit eines Systems. 	
\item 	Technical Debt \\ Technische Schuld oder Technische Schulden (engl. technical debt) ist eine in der Informatik gebräuchliche Metapher für die möglichen Konsequenzen schlechter technischer Umsetzung von Software. Unter der Technischen Schuld versteht man den zusätzlichen Aufwand, den man für Änderungen und Erweiterungen an schlecht geschriebener Software im Vergleich zu gut geschriebener Software einplanen muss.
\item 	Tester \\ Eine Person die einen Test durchführt.
\item 	Testfall \\ Ein Testfall beschreibt einen elementaren, funktionalen Softwaretest, der der Überprüfung einer z. B. in einer Spezifikation zugesicherten Eigenschaft eines Testobjektes dient. 
\item 	Testkandidat \\ Ein zu testendes Objekt. (Klasse, Methode, etc.)
\item 	Teststrategie \\ Eine Strategie wie ein Test durchgeführt werden soll. Sie beschäftigen sich u.a. mit der Frage, wie mit einer möglichst geringen Anzahl von Testfällen eine große Testabdeckung zu erreichen ist.
\item 	Testziel \\ Ein Grund oder Zweck für den Entwurf und die Ausführung von Tests.
\item 	Transient \\ Gegenteil von persistenz
\item 	UML \\ Die Unified Modeling Language, kurz UML, ist eine grafische Modellierungssprache zur Spezifikation, Konstruktion und Dokumentation von Software-Teilen und anderen Systemen
\item 	Unterdomäne \\ Ein Teilgebiet eines großen Ganzen.
\item 	Usecase \\ Ein Anwendungsfall (engl. use case) bündelt alle möglichen Szenarien, die eintreten können, wenn ein Akteur versucht, mit Hilfe des betrachteten Systems ein bestimmtes fachliches Ziel (engl. business goal) zu erreichen. Er beschreibt, was inhaltlich beim Versuch der Zielerreichung passieren kann und abstrahiert von konkreten technischen Lösungen. Das Ergebnis des Anwendungsfalls kann ein Erfolg oder Fehlschlag/Abbruch sein.
\item 	Validierung \\ Im Bereich der Softwarequalitätssicherung wird unter Validierung (Validation) die Prüfung der Eignung beziehungsweise der Wert einer Software bezogen auf ihren Einsatzzweck verstanden. Die Eignungsprüfung erfolgt auf Grundlage vorher aufgestellten Anforderungsprofils und kann sowohl technisch als auch personell geschehen.

Umgangssprachlich formuliert wird die Frage „Wird das richtige Produkt entwickelt?“ beantwortet. Es wird also die Effektivität der Entwicklung sichergestellt. Im Gegensatz dazu steht die Verifikation, ein Prozess, der für ein Programm oder ein System sicherstellt, dass es zu einer Spezifikation „konform“ ist („Ist das System richtig gebaut?“).
\item 	Verifikation \\ In der Informatik und Softwaretechnik versteht man unter Verifikation einen Prozess, der für ein Programm oder ein System sicherstellt, dass es zu einer Spezifikation „konform“ ist („Ist das System richtig gebaut?“). Im Gegensatz dazu steht die Validierung, d. h. die dokumentierte Plausibilisierung, dass ein System die Anforderungen in der Praxis erfüllt („Funktioniert das System richtig?“).
\item 	Verteilte Objektarchitekturen \\ Verteilte Objektarchitekturen haben die Aufgabe, die Kommunikation zwischen Objekten auf räumlich getrennten Computersystemen wie eine lokale Kommunikation erscheinen zu lassen. Dazu werden auf der Seite des Aufrufers meist automatisch lokale Stellvertreterobjekte erzeugt, die Operationsaufrufe annhemen, in Datenpakete umwandeln und über das Netz schicken. Auf der Seite des aufgerufenen Objekts werden die empfangenen Datenpakete wieder zu einem lokalen Operationsaufruf zusammengesetzt (CORBA, RMI)
\item 	Verteilte System \\ Ein Verteiltes System ist ein Zusammenschluss unabhängiger Computer, die sich für den Benutzer als ein einziges System präsentieren.
\item 	V-Modell \\ Das V-Modell ist ein Vorgehensmodell in der Softwareentwicklung, bei dem der Softwareentwicklungsprozess in Phasen organisiert wird. Neben diesen Entwicklungsphasen definiert das V-Modell auch das Vorgehen zur Qualitätssicherung (Testen) phasenweise.
\item 	Whitebox-Test \\ Der Begriff White-Box-Test bezeichnet eine Methode des Software-Tests, bei der die Tests mit Kenntnissen über die innere Funktionsweise des zu testenden Systems entwickelt werden. Im Gegensatz zum Black-Box-Test ist für diesen Test also ein Blick in den Quellcode gestattet, d. h. es wird am Code geprüft.
\item 	Workaround \\ Ein Workaround (englisch: um etwas herum arbeiten) ist ein Umweg zur Vermeidung eines bekannten Fehlverhaltens eines technischen Systems. Es ist ein Hilfsverfahren, das das eigentliche Problem nicht behebt, sondern mit zusätzlichem Aufwand seine Symptome umgeht.
\item 	XML \\ Die Extensible Markup Language, abgekürzt XML, ist eine Auszeichnungssprache zur Darstellung hierarchisch strukturierter Daten in Form von Textdateien. XML wird u. a. für den plattform- und implementationsunabhängigen Austausch von Daten zwischen Computersystemen eingesetzt, insbesondere über das Internet.
\item 	Zustandsdiagramm \\ Das Zustandsdiagramm stellt einen endlichen Automaten grafisch dar und wird benutzt, um entweder das Verhalten eines Systems oder die zulässige Nutzung der Schnittstelle eines Systems zu spezifizieren.
\end{itemize}