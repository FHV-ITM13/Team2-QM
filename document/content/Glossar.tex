\chapter{Glossar}
\begin{itemize}
\item Software
\item Software-Engineering
\item Informatik
\item Softwareprozess
\item Software-Validierung (o	Ob die Software für den Kunden einen Wert hat)
\item Software-Verifizierung (o	Übereinstimmung der Software mit den Spezifikationen)
\item Qualität
\item Technisches computer-basiertes System
\item Soziotechnisches System
\item Funktionale Anforderungen
\item Nicht-Funktionale Anforderungen
\item Technical Debt
\item Verifikation
\item Validierung
\item ESSENCE Kernel
\item Data Dictionary
\item Scope: was man wirklich implementieren will
\item Kontext: das ganze um den Scope herum.
\item Domäne
\item Szenarien
\item Anforderung
\item Anforderungsreview
\item Stakeholder
\item Systemgrenzen
\item Kontext
\item Scope
\item Usecase
\item UML
\item Architekturdiagramm
\item Datenflussdiagramm
\item Prozessdiagramm
\item Zustandsdiagramm
\item Sequenzdiagramm
\item ER-Diagramm
\item XML
\item OWL
\item transients 
\item persistents
\item Objektorientierten Modellierung
\item Datenfluss
\item Ereignis
\item Aggregationsbeziehungen
\item JSP
\item V-Modell
\item RUP
\item Qualität
\item mandatory requirements was muss das fertige System können
\item Qualitative Anforderung: es hat die Eigenschaft oder hat sie nicht
\item Quantitative Anforderung: ist eine messbare Eigenschaft z.B. die Antwortzeit beträgt max. 1 Sekunde
\item Pflichtenheft
\item Kunden
\item Manager
\item Entwickler
\item Tester
\item Klassifizierungen
\item Normen
\item Standards
\item Systemzuverlässigkeit
\item Systemwiederherstellungszeit
\item Systemverfügbarkeit
\item Risikomanagement
\item Workaround
\item Sicherheitslücke
\item Hardware

\end{itemize}