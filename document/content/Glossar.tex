\chapter{Glossar}
\begin{itemize}
\item 	Aggregationsbeziehungen\\
Falls Objekte einer Klasse Bestandteil von einem anderen Objekt sind, leigt eine Aggregationsbeziehung vor.
\item 	Anforderung
\item 	Anforderungsreview\\
Bei einem Anforderungsreview werden die Anforderungen von allen Stakeholdern sowie Systemarchitekten und Softwareentwickler gemeinsam diskutiert. Dadurch werden Konflikte, Fehler und Versäumnisse aufgedeckt. Es dient der Validierung der Anforderungen.
\item 	Architekturdiagramm\\
Bildet die Architektur einer Software ab.
\item 	Blackbox-Test (Tests.pdf S. 14)\\
Test gegen die Spezifikation ohne Kenntnis des Codes.
\item 	Bug\\
Bezeichnet ein Fehlverhalten von Computerprogrammen.
\item 	Client/Server-Architektur\\
Unter der Client-Server-Architektur versteht man eine kooperative Informationsverarbeitung, bei der die Aufgaben zwischen Programmen auf verbundenen Rechnern aufgeteilt werden.
\item 	Data Dictionary\\
Ein Katalog von Metadaten, der die Definitionen und Darstellungsregeln für alle Anwendungsdaten und die Beziehungen zwischen den verschiedenen Datenobjekten enthält, damit der Datenbestand redundanzfrei und einheitlich strukturiert wird. Es ist ein Anwendungsfall eines spezifischen Datenmodells.
\item 	Dienstorientierte Architekturen\\
Serviceorientierte Architektur (SOA, englisch service-oriented architecture), auch dienstorientierte Architektur, ist ein Architekturmuster der Informationstechnik
\item 	Emulator\\
Nachbildung wesentlicher Verhaltensaspekte eines Systems durch ein anderes.
\item 	Endknoten\\
Knoten ein einem Graph von dem man nicht mehr weg kommt.
\item 	endliche Automaten\\
Ein Modell eines Verhaltens, bestehend aus Zuständen, Zustandsübergängen und Aktionen.
\item 	Entwickler\\
Als Entwickler wird die Person bezeichnet die die Software entwickelt.
\item 	ER-Diagramm\\
Die Darstellung eines Entity-Relationship-Modells
\item 	Ereignis\\Unter einem Ereignis (englisch event) versteht man in der ereignisorientierten Programmierung eine Begebenheit, die – über ein Event-Handler-Programm – eine Aktion und (in deren Folge) ggf. eine Zustandsveränderung auslöst. Diese Ereignisse können Benutzereingaben (Mausklick, Taste, Spracheingabe, Geräteanschluss, …) oder Systemereignisse (Zeitpunkt, Fehler, Datenveränderung, Sensor, …) sein.
\item 	ESSENCE Kernel\\
Ein Bereich aus der SEMAT Initiative
\item 	Funktionale Anforderungen\\
Eine funktionale Anforderung legt fest, was das Produkt tun soll. Ein Beispiel:
\\
„Das Produkt soll den Saldo eines Kontos zu einem Stichtag berechnen.“
\item 	Hardware\\
Die mechanische und elektronische Ausrüstung eines Systems.
\item 	Informatik\\Informatik ist die „Wissenschaft der systematischen Verarbeitung von Informationen, insbesondere der automatischen Verarbeitung mit Hilfe von Digitalrechnern“.
\item 	JSP\\Jackson structured programming (JSP) ist eine Methode zum strukturierten Entwurf von Computerprogrammen.
\item 	Kunden\\Ein Kunde ist eine Person oder eine Institution, die ein offensichtliches Interesse am Vertragsschluss zum Zwecke des Erwerbs eines Produkts oder einer Dienstleistung gegenüber einem Unternehmen oder einer Institution zeigt.
\item 	Manager\\
Ein Manager ist eine Person im Anstellungsverhältnis, welche Managementaufgaben in einer Organisation wahrnimmt. Die wichtigsten Managementaufgaben sind Planung, Organisation, Führung und Kontrolle.
\item 	mandatory requirements\\
Was muss das fertige System können
\item 	Marketingstrategen
\\Personen die Marketingstrategien umsetzen.
\item 	Mehrprozessor-Architekturen\\
Computersysteme mit zwei oder mehr Prozessoren
\item 	Messwert\\Ein Messwert ist der Wert einer Messgröße, der von einem Messgerät oder einer Messeinrichtung geliefert wird.
\item 	Nicht-Funktionale Anforderungen\\
Eine nichtfunktionale Anforderung (englisch non-functional requirement, NFR) legt fest, welche Eigenschaften das Produkt haben soll. Ein Beispiel:
\\
„Das Produkt soll dem Anwender innerhalb von einer Sekunde antworten.“
\item 	Normen\\Normen sind anerkannte Regeln in der Technik. 
\item 	Objektorientierten Modellierung\\Die objektorientierte Modellierung bildet einen Ansatz zur Analyse und Entwicklung von Systemen, der wesentlich auf den Konzepten „Objekt“, „Klasse“ und „Vererbung“ beruht.
\item 	OWL\\
Die Web Ontology Language (kurz OWL) ist eine Spezifikation des World Wide Web Consortiums (W3C), um Ontologien anhand einer formalen Beschreibungssprache erstellen, publizieren und verteilen zu können. Es geht darum, Termini einer Domäne und deren Beziehungen formal so zu beschreiben, dass auch Software die Bedeutung verarbeiten („verstehen“) kann.
\item 	Peer-to-Peer-Architekturen\\
P2P ist ein netzwerkbasiertes Modell für Anwendungen, wo Computer Ressourcen und
Dienste über den direkten Austausch teilen. Alle Computer können Dienste beanspruchen
oder zur Verfügung stellen, worauf besonders acht darauf gelegt wird, dass keine
Abhängigkeit von zentralen Servern besteht.
\item 	persistent\\Persistenz ist in der Informatik der Begriff, der die Fähigkeit bezeichnet, Daten (oder Objekte) oder logische Verbindungen über lange Zeit (insbesondere über einen Programmabbruch hinaus) bereitzuhalten
\item 	Pflichtenheft\\Das Pflichtenheft beschreibt in konkreter Form, wie der Auftragnehmer die Anforderungen des Auftraggebers zu lösen gedenkt – das sogenannte wie und womit.
\item 	Prozessdiagramm\\Die Darstellung eines Prozessablauf.
\item 	Qualität\\
Qualität wird laut der Norm EN ISO 9000:2005 (der gültigen Norm zum Qualitätsmanagement), als „Grad, in dem ein Satz inhärenter Merkmale Anforderungen erfüllt“, definiert. Die Qualität gibt damit an, in welchem Maße ein Produkt (Ware oder Dienstleistung) den bestehenden Anforderungen entspricht.
\item 	Qualitative Anforderung: es hat die Eigenschaft oder hat sie nicht
\item 	Qualitätsmanagement\\
Qualitätsmanagement (QM) bezeichnet alle organisatorischen Maßnahmen, die der Verbesserung der Prozessqualität, der Leistungen und damit den Produkten jeglicher Art dienen. 
\item 	Quantitative Anforderung: ist eine messbare Eigenschaft z.B. die Antwortzeit beträgt max. 1 Sekunde
\item 	redundante Hardware\\
Bei Hardware heit Redundanz, dass mehrere Geräte vorhanden sind (obwohl das System auch mit einem Bruchteil auskommen würde), die zu mehr Stabilität und/oder Geschwindigkeit führen. z.B. RAID
\item 	Reengineering\\
In diesem Dokument bedeutet Reengineering die Umwandlung eines alten System in ein neues, ganz besonders im Blick auf die verwendete Programmiersprache.
\item 	Regressionstest \\
Test, der neben der erwarteten Funktionalität sicherstellen soll, dass die Software frei ist von allen bislang bereits beseitigten Fehlern
\item 	Review\\
Mit dem Review werden Arbeitsergebnisse der Softwareentwicklung manuell geprüft. Jedes Arbeitsergebnis kann einer Durchsicht durch eine andere Person unterzogen werden. Der oder das Review ist eine statische Testmethode und gehört in die Kategorie der analytischen Qualitätssicherungsmaßnahmen.

In Anlehnung an die IEEE-Norm 729 ist das Review ein mehr oder weniger formal geplanter und strukturierter Analyse- und Bewertungsprozess, in dem Projektergebnisse einem Team von Gutachtern präsentiert und von diesem kommentiert oder genehmigt werden.
\item 	Risikomanagement\\
Risikomanagement umfasst sämtliche Maßnahmen zur systematischen Erkennung, Analyse, Bewertung, Überwachung und Kontrolle von Risiken.
\item 	RUP\\Der Rational Unified Process (RUP) ist ein kommerzielles Produkt der Firma Rational Software, die seit 2003 Teil des IBM-Konzerns ist. Es beinhaltet sowohl ein Vorgehensmodell zur Softwareentwicklung als auch die dazugehörigen Softwareentwicklungsprogramme. IBM entwickelt den RUP und die zugehörige Software weiter. Die 9. Version ist die derzeit (2006) aktuelle Version. Der RUP benutzt die Unified Modeling Language (UML) als Notationssprache. Der RUP wurde von Philippe Kruchten in seiner Urform erstmals 1998 vorgestellt.
\item 	Sequenzdiagramm\\
Sequenzdiagramme beschreiben die Kommunikation zwischen Objekten in einer bestimmten Szene. Es wird beschrieben welche Objekte an der Szene beteiligt sind, welche Informationen (Nachrich­ten) sie austauschen und in welcher zeitlichen Reihenfolge der Informationsaustausch stattfindet. Sequenzdiagramme enthalten eine implizite Zeitachse. Die Zeit schreitet in einem Diagramm von oben nach unten fort. Die Reihenfolge der Pfeile in einem Sequenz­diagramm gibt die zeitliche Reihenfolge der Nachrichten an.
\item 	Sicherheitslücke\\Eine Sicherheitslücke ist im Gebiet der Informationssicherheit ein Fehler in einer Software, durch den ein Programm mit Schadwirkung (Malware) oder ein Angreifer in ein Computersystem eindringen kann.
\item 	Simulator\\Vereinfachtes Modell eines Systems zum Zweck der Analyse dieses Systems.
\item 	Software\\
Software ist ein Sammelbegriff für ausführbare Programme und die zugehörigen Daten. Sie dient dazu, Aufgaben zu erledigen, indem sie von einem Prozessor ausgewertet wird und so softwaregesteuerte Geräte in ihrer Arbeit beeinflusst.
\item 	Softwarearchitekten \\
Ein Begriff für Menschen die im Bereich der Softwaretechnik an der Softwarearchitektur und deren Implementierung arbeiten.
\item 	Software-Engineering\\
Beschäftigt sich mit der Herstellung bzw. Entwicklung von Software, der Organisation und Modellierung der zugehörigen Datenstrukturen und dem Betrieb von Softwaresystemen
\item 	Softwareprozess\\Software-Prozesse beschreiben den Ablauf der Entwicklung und der Pflege von Software.
\item 	Software-Validierung \\  Ob die Software für den Kunden einen Wert hat)
\item 	Software-Verifizierung \\ Übereinstimmung der Software mit den Spezifikationen)
\item 	Soziotechnisches System\\Unter einem soziotechnischen System versteht man eine organisierte Menge von Menschen und Technologien, welche in einer bestimmten Weise strukturiert sind, um ein spezifisches Ergebnis zu produzieren.
\item 	Spezifikation\\Eine Spezifikation  ist eine formalisierte Beschreibung eines Produktes, eines Systems oder einer Dienstleistung. Ziel der Spezifikation ist es, Merkmale zu definieren und zu quantifizieren (Toleranzwerte), mit denen das Werk oder die Dienstleistung des Auftragnehmers bei der Übergabe an den Auftraggeber bzw. Käufer geprüft und durch den Auftraggeber abgenommen werden kann, bzw. nach der der Auftragnehmer bzw. Verkäufer die Bezahlung fordern kann, wenn die Merkmale der Spezifikation erreicht wurden. Die Spezifikation enthält in der Regel für jede spezifizierte Eigenschaft eine präzise Referenz zu der anzuwendenden Prüfmethode für das jeweilige Merkmal.
\item 	Stakeholder
\item 	Standards
\item 	Startknoten
\item 	Strategie
\item 	SuT
\item 	Systemarchitekten
\item 	Systemgrenzen
\item 	Systemverfügbarkeit
\item 	Systemwiederherstellungszeit
\item 	Systemzuverlässigkeit
\item 	Szenarien
\item 	Technical Debt
\item 	Technisches computerbasiertes System
\item 	Tester
\item 	Testfall
\item 	Testkandidat
\item 	Teststrategie
\item 	Testziel
\item 	transient
\item 	UML
\item 	Unterdomäne
\item 	Usecase
\item 	Validierung
\item 	Verifikation
\item 	Verteilte Objektarchitekturen
\item 	Verteilte Systeme
\item 	V-Modell
\item 	Webapplikationen
\item 	Whitebox-Test(Tests.pdf S. 14)
\item 	Workaround
\item 	XML
\item 	Zustandsdiagramm
\end{itemize}