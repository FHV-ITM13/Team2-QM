\chapter{LV 4 am 25.04.2014}
\section{Dokumentation und Verwaltung}
\subsection{Pflichtenheft}
Das Pflichtenheft dient zur Zusammenstellung der vollständigen und detaillierten Benutzer- und Systemanforderungen. Wichtig ist, dass das Pflichtenheft vollständig ist, wobei dieses Ziel nur sehr schwer zu erreichen ist. Bei Projekte der Europäischen Union ist es sogar nicht mehr erlaubt Änderungen am Pflichtenheft nach Vertragsabschluss durchzuführen. Dies ist durch ein Wettbewerbsgesetz geregelt. 
\linebreak 
Ein Pflichtenheft kann aus mehr als einem Dokument bestehen und wird  für  Kunden, Manager, Entwickler, Tester, Warter und die Juristen geschrieben. 
\linebreak
Ein beliebter Trick in diversen Ländern ist es, das Pflichtenheft via E-Mail abzuändern. Wenn diese Änderung allerdings nicht innerhalb von 14 Tagen widerrufen wird, dann ist sie Bestandteil vom Vertrag und muss entwickelt werden. Das gleiche gilt für Besprechungen, die schriftlich Protokolliert werden. Sollten diese Änderungen dann nicht berücksichtigt werden, kann ein Rechtsstreit eintreten.
\linebreak
Im Pflichtenheft erfolgt eine Klassifizierungen der Informationen hinsichtlich der Anforderungen (mandatory requirements), Wünsche (optional requirements), informative Bestandteile (dienen zum besseren Verständnis, sind unverbindliche Informationen) und Warnungen (Abbildungen, Tabellen, Anhänge als informative Bestandteile). Um informative Bestandteile wie ein Anhang zu einem vertraglicher Bestandteil zu machen, muss es mit Normative gekennzeichnet werden.
\linebreak
Der Aufbau vom Pflichtenheft kann auch nach vordefinierten Norme und Standards erfolgen. Ein Beispiel für eine solche Norm ist die IEEE/ANSI 830-1998 Norm. Wichtig ist das auch die Normen und Standards von Referenzen zum Bestandteil des Vertrages werden.

\section{Spezifikation kritischer Systeme}
Kritische Systeme sind jene, bei deren Benutzung durch das Auftreten eines Fehlers Gefahr besteht - zum Beispiel ein GPS-System oder ein Defibrillator in der Medizintechnik.

\subsection{Anforderungen an kritische Systeme}
Funktionalen Anforderungen unterstützen Definitionen von  Funktionen zur Fehlerprüfung, Funktionen zur Wiederherstellung im Fehlerfall und Aspekte zum Schutz gegen Systemausfälle. Zu den nicht funktionale Anforderungen zählen die Systemzuverlässigkeit und die Systemverfügbarkeit. Es werden auch Negativanforderungen festgelegt, welche jene Verhalten beschreiben, das ein System auf keinen Fall zeigen darf beschreiben. Eine Negativanforderung wäre zum Beispiel das ein Flugzeug nicht schneller als 800km/h fliegen darf.


\subsection{Risikomanagement}
Beim Risikomanagement werden Gefahren bestimmt, klassifiziert und versucht deren Risiko zu verringern.
\linebreak
Prinzipiell gelten Murphy's Gesetze:
"Alles, was schiefgehen kann, wird irgendwann einmal schiefgehen."
"Alles, was schiefgehen kann, wird genau dann schiefgehen, wenn es den maximalen Schaden verursacht."
oder Wenzels Korollar dazu
"Murphy war ein Optimist."
\linebreak
Das Ziel beim Risikomanagement ist es, Risiken zu erkennen und dessen Auswirkungen zu minimieren. Risiken entstehen prinzipiell dort, wo Systemkomponenten aufeinandertreffen oder die Umwelt Einflüsse auf das System hat. Entgegenwirken kann man durch erfahrene Entwickler, Berater oder Fachleute des Anwendungsgebiets (siehe witziges Beispiel Ölschraube bei Scania).
\linebreak
Bei der Risikoanalyse und der Risikoklassifizierung spielen die Begriffe Unfallwahrscheinlichkeit, Schadenswahrscheinlichkeit und Schadenshöhe eine wichtige Rolle. Durch die Klassifizierung kann man  abschätzen, ob ein Risiko vernachlässigbar ist oder ob es unannehmbar ist. Unannehmbare Risiken verursachen große Schaden, haben eine hohe Schadenswahrscheinlichkeit und sollten unter allen Umständen versucht werden zu vermeiden.
\linebreak
Um herauszufinden woher ein Risiko kommt wird es zerlegt um die Ursache für das Risiko zu entdecken. Hierfür gibt es viele verschiedene Techniken wie Reviews, Checklisten, Petri-Netze, Fehlerbaum sowie einige andere Techniken.
\linebreak
Um die Risiken minimieren zu können, werden in der Praxis die Techniken zur Risikovermeidung-, Risiko-Erkennung- und die Risikobeseitigungen- sowie die Schadensbegrenzungsstrategie kombiniert.

\subsection{Betriebssicherheit}
Wichtig ist es das System in Hinsicht auf dessen Betrieb zu sichern. Vor allem Schutzmechanismen und Ausfallsysteme sind von großer Bedeutung. Auch die Betriebssicherheit im Laufe des Produktlebenszyklus ist zu beachten. Um Fehler in diesem Bereich klassifizieren zu können, gibt es verschiedene Klassen. Klasse 1 bis Klasse 5 werden dabei in der Software Industrie als Standard angesehen.

\begin{itemize}
\item Klasse 1: Fehler die andere Anwendungen betreffen können.
\item Klasse 2: Fehler die Benutzer der selben Anwendung betreffen können.
\item Klasse 3: Funktion steht ohne Workaround nicht zur Verfügung.
\item Klasse 4: Funktion steht nur mittels Workaround zur Verfügung.
\item Klasse 5: Vierbesserungswünsche 
\end{itemize}

\subsection{Systemsicherheit}
Bei der Systemsicherheit spielen vor allem die Bedrohungen eine wichtige Rolle. Die Bedrohungen können dabei in Kategorien unterteilt werden. (Direkte Bedrohungen, Indirekte Bedrohungen, etc.). Vor allem die Monopolartige Stellung einiger Unternehme führt dazu das Sicherheitslücken im großen Stil auftreten können. Zudem gibt es keine 100% Sicherheit, es gibt nur gute Sicherheitsmethoden. Bei der Systemsicherheit geht es vorallem um das Vermeiden von Schäden durch unberechtigte Personen. 

\subsection{Zuverlässigkeit}
Damit ein System zuverlässig ist, müssen sowohl die Hardware als auch die Software und die Benutzer zuverlässig sein. Die Zuverlässigkeit einer Software ist messbar. Dabei werden Messwerte bezüglich Systemfehler, Zeit zwischen Systemfehlern und ähnlichem verwendet. Vor allem die Zeit zwischen Systemfehler und Wiederverfügbarkeit des Systems ist wichtig für die Aussage über die Zuverlässigkeit eines Systems.
\linebreak
Die Vorgehensweise bei der Zuverlässigkeitsspezifikation  ist es die möglichen Systemfehler zu ermitteln, deren Auswirkungen zu analysieren und mögliche Vorkehrungen zur Beseitigung dieser Risiken zu realisieren. Besonders wichtig ist es allerdings zu berücksichtigen, dass extreme Zuverlässigkeit nicht testbar ist.

\subsection{Formale Spezifikationsmethoden}
In die formale Spezifikationsmethoden wird schon seit langer Zeit gearbeitet. Der Durchbruch dieser Technik blieb allerdings bis heute aus. Die Gründe dafür sind neben der Entstehung von anderen Software-Engineering-Methoden auch die Marktveränderungen in diesem Bereich und die beschränkte Einsetzbarkeit der Methode. Allerdings ist diese Methode vor allem in den Bereichen des Luftverkehrs, der Raumfahrt und in der Medizintechnik weit verbreitet. Die Nutzung in Softwareprozessen kann entweder mittels einem Linearen Prozess oder einem Parallelen Prozess erfolgen. Die Spezifikation von System-Schnittstellen kann mit unter allerdings recht Umfangreich werden. Auf Grund dieses Problems entsteht hier eine Chance für den Model Driven Architekuture Ansatz.
\linebreak
Formale Spezifikationstechniken können als gute Ergänzungen zu informellen Techniken betrachtet werden. Vor allem im Kostenvergleich können sie überzeugen. 