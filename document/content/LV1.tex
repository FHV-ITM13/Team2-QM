\chapter{LV 1 am 21.02.2014}
\section{Organisatorisches}
Zu Beginn dieser Vorlesung wurden zuerst die organisatorischen Randbedingungen für diese Lehrveranstaltung geklärt.\\
Dabei wurde entschieden, dass anstelle einer Prüfung ein Lerntagebuch in Gruppen zu max. drei Personen erstellt wird. Das Lerntagebuch enthält zusätzlich ein Glossar mit den wichtigsten Begriffsdefinitionen rundum das Thema Softwareprozesse und Softwarequalität.

\section{Motivation des Software-Engineering}
Als Einstieg in das Thema dienten ein paar nicht ganz ernst gemeinte "Gesetze". Diese Gesetze sollten uns verdeutlichen, dass Software-Engineering nicht so trivial und einfach ist, wie es oft den Anschein hat. Wie schon die Folgerung von Kleinbrunner besagt
\begin{quote}
"Wenn eine Programmieraufgabe leicht aussieht, ist sie schwer."
\end{quote}
Es wurden unter anderem die "Gesetze" von Gutterson, Farvour, Brunk, Munbright und vielen anderen vorgestellt.
\\\\
Um die Bedeutung von qualitativ hochwertiger Software zu verdeutlichen, stellte der Vortragende einige Probleme in bekannten Softwareprojekten wie beispielsweise der Verlust einer Segelyacht mit Crew vor der mexikanischen Pazifikküste aufgrund eines Softwarequalitätsfehlers.
\\\\
Abschließend erhielten wir die Aufgabe bis zur nächsten Vorlesung ein Programm zu entwickeln zur Berechnung der reellen Lösungen einer quadratischen Gleichung. Ziel bei dieser Aufgabe ist es auf die Qualität des Codes zu achten. 